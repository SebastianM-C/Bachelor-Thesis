\documentclass[../thesis.tex]{subfiles}

%!TeX spellcheck = en-GB

\theoremstyle{plain}
\newtheorem*{theorem*}{Theorem}

\begin{document}

\chapter{Classical chaos}
\label{chap:classical-chaos}

\section{Fundamental notions}

Classical mechanics can be reformulated in a more elegant way which does not
require the computation of individual forces and instead uses a single function to
describe the entire system, namely the \emph{Lagrangian}.
The Lagrangian, denoted with \(\mathcal{L}\), is a function of the generalised
coordinates \(q_1, \dotsc, q_n\), their derivatives with respect to time
\(\dot{q_1}, \dotsc, \dot{q_n}\) and optionally the time \(t\), where \(n\)
is the number of degrees of freedom.

The principles of Newtonian mechanics are replaced by the a variational principle
\(\var{S} = 0\), where
\begin{equation}
  \label{eq:action}
  S = \int_{t_1}^{t_2} \mathcal{L}(q_1, \dotsc, q_n, \dot{q_1}, \dotsc, \dot{q_n},t) \dd{t}
\end{equation}
is called the \emph{action}.
This principle, also called \emph{Hamilton's principle} or \emph{the principle of least action}
is equivalent with the Euler-Lagrange equations
\[
  \pdv{\mathcal{L}}{q_k} - \dv{t}\pdv{\mathcal{L}}{\dot{q_k}} = 0,
\]
where \(k=1,\dotsc, n\).
Thus we can obtain the equations of motion provided that we know the initial
conditions \(q_1(t_0), \dotsc, q_n(t_0)\) and \(\dot{q_1}(t_0), \dotsc, \dot{q_n}(t_0)\)
at \(t=t_0\) by solving the set of \(n\) second order differential equations.

An other formalism for classical mechanics which is very useful in theoretical physics is
the Hamiltonian formalism. In this formalism instead of the \mbox{Lagrangian} we use the
\emph{Hamiltonian}, \(\mathcal{H}\), which is a function of the generalised coordinates
\(q_1, \dotsc, q_n\), the generalised momenta \(p_1, \dotsc, p_n\) and optionally
the time \(t\). The generalised momenta are given by
\[
  p_k = \pdv{\mathcal{L}}{\dot{q_k}}.
\]

The Hamiltonian can be obtained from the Lagrangian through a \emph{Legendre transformation} of
\(\mathcal{L}\)
\begin{equation}
  \label{eq:legendre-tr}
  \mathcal{H}(q_1, \dotsc, q_n, p_1, \dotsc, p_n, t)
  = \sum_{k=0}^n p_k\dot{q_k} - \mathcal{L}(q_1, \dotsc, q_n, \dot{q_1}, \dotsc, \dot{q_n}, t).
\end{equation}

The Euler-Lagrange equations are replaced by Hamilton's equations
\begin{align*}
  \dot{q_k} = \pdv{\mathcal{H}}{p_k}, &&
  \dot{p_k} = -\pdv{\mathcal{H}}{q_k}.
\end{align*}
We now have \(2n\) first order ordinary differential equations.

\section{Canonical transformations}

This section follows mainly the steps described in~\cite{Hand1998}.
In the Hamiltonian formulation of classical mechanics \emph{canonical transformations}
represent all the transformations of the canonically conjugate variables from
$q$ and $p$ to $Q$ and $P$ such that the structure of the Hamilton equations remains intact.

The Lagrange equations of a system are invariant to a transformation of the
Lagrange function such that:
\[
  \mathcal{L'}(Q, \dot{Q}, t) = \mathcal{L}(q, \dot{q}, t) - \dv{t}F(q, Q, t).
\]
Integrating we obtain
\[
  \int_{t_1}^{t_2} \mathcal{L'} \dd{t} = \int_{t_1}^{t_2} \mathcal{L} \dd{t}
  - \left( F(q, Q, t_2) - F(q, Q, t_1) \right).
\]
The variation of the above equation is given by
\[
  \var{\int_{t_1}^{t_2} \mathcal{L'} \dd{t}} = \var{\int_{t_1}^{t_2} \mathcal{L} \dd{t}}
  - \var{F(q, Q, t_2)} + \var{F(q, Q, t_1)}.
\]
Fixing the variations at the end points to be $0$, we get
\[
  \var{\int_{t_1}^{t_2} \mathcal{L'} \dd{t}} = \var{\int_{t_1}^{t_2} \mathcal{L} \dd{t}}.
\]
So indeed the two Lagrangian descriptions of the system are identical since if the
\mbox{Hamilton} principle holds in the first description it also holds in the second.

A function \(F(q, Q, t)\) satisfying the above property and the condition
\(\pdv{F}{q}{Q} \neq 0\) is called a \emph{generating function}.
We can classify the generating functions in four possible types
as a function of their variables.
The conjugate momenta corresponding to $q$ and $Q$ are given by
\[
  p = \pdv{\mathcal{L}}{\dot{q}} \text{ and } P = \pdv{\mathcal{L'}}{\dot{Q}}.
\]
In order to express those as a function of \(F(q, Q, t)\) we use the fact
that \(\dot{q}\) is a cyclic coordinate for \(\mathcal{L'}\)
\[
  \pdv{\mathcal{L'}}{\dot{q}} = \pdv{\mathcal{L}}{\dot{q}} -
  \pdv{\dot{q}} \left( \dv{t}F(q, Q, t) \right) =
  \pdv{\mathcal{L}}{\dot{q}} - \pdv{F}{q} = 0,
\]
where
\[
  \dv{t}F(q, Q, t) = \pdv{F}{q} \dot{q} + \pdv{F}{Q} \dot{Q} + \pdv{F}{t}.
\]
Thus we obtain
\[
  p = \pdv{\mathcal{L}}{\dot{q}} = \pdv{F}{q}
\]
and
\[
  P = \pdv{\mathcal{L'}}{\dot{Q}} = - \pdv{F}{Q}.
\]

To find the Hamiltonian corresponding to the new coordinates $Q$ and $P$ we
compute the Legendre transformation of \(\mathcal{L'}\)
\[
  \mathcal{H'}(Q, P, t) = P\dot{Q} - \mathcal{L'}(Q, \dot{Q}, t)
\]
Thus
\[
  \mathcal{H'}(Q, P, t) = P\dot{Q} - \mathcal{L} + \dv{F}{t} =
  P\dot{Q} - \mathcal{L} + \pdv{F}{q} \dot{q} + \pdv{F}{Q} \dot{Q} + \pdv{F}{t} =
  p\dot{q} - \mathcal{L} + \pdv{F}{t}
\]
and so
\begin{equation}
  \label{eq:canonical-transf}
  \mathcal{H'}(Q, P, t) = \mathcal{H}(q(Q,P), p(Q, P), t) + \pdv{F(q(Q,P), Q, t)}{t}.
\end{equation}

As mentioned above, there are four types of generating functions. They can be obtained
by successive Legendre transformations which switch between the conjugate coordinates and
they represent the same canonical transformation. We denote the above mentioned
\(F(q, Q, t) \equiv F_1(q, Q, t)\). The other types of generating functions will be denoted
\(F_2(q, P, t), F_3(p, Q, t), F_4(p, P, t)\). Thus we define
\begin{align*}
  F_2(q, P, t) &= F_1(q, Q, t) + PQ &&
  p = \pdv{F_2}{q}, & Q = \pdv{F_2}{P} \\
  F_3(p, Q, t) &= F_1(q, Q, t) - pq &&
  q = - \pdv{F_3}{p}, & P = \pdv{F_3}{Q} \\
  F_4(p, P, t) &= F_1(q, Q, t) - pq + PQ &&
  q = - \pdv{F_4}{p}, & Q = \pdv{F_4}{P}.
\end{align*}

The \emph{Poisson bracket} of two arbitrary functions with respect to the canonical coordinates
$q$ and $p$ is defined as
\[
  {\left[F, G\right]}_{q,p} = \sum_{k=1}^n \left( \pdv{F}{q_k} \pdv{G}{p_k} -
                                                  \pdv{F}{p_k} \pdv{G}{q_k} \right).
\]
We can use the Poisson bracket to express Hamilton's equations more elegantly
\begin{align*}
  \dot{q}_k = [q_k,H], && \dot{p}_k = [p_k,H].
\end{align*}

Expressed in this form, the Hamilton equations have a close resemblance to
Heisenberg's form for the equations of motion in the Heisenberg picture of
quantum mechanics. The Poisson bracket structure of classical mechanics is
fundamental and provides a link with quantum mechanics, being in correspondence
with the commutator algebra~\cite{Dirac1967}.

As with the above discussion we will limit to the case of one degree of freedom,
but the results are easily generalisable to $n$ degrees of freedom.

If $Q$ and $P$ are a set of conjugate variables obtained by a canonical transformation
from $q$ and $p$, then
\[
  {\left[F', G'\right]}_{Q,P} = {\left[F, G\right]}_{q,p},
\]
where $F'$ and $G'$ are the transformed functions.
As a consequence, if we choose \(F=Q\) and \(G=P\)
\[
  {\left[Q, P\right]}_{Q,P} = {\left[Q(q,p), P(q,p)\right]}_{q,p} = 1
\]
This relation represents a necessary and sufficient condition for the transformation from
$q,p$ to $Q,P$ to be canonical.
Preserving the structure of Hamilton's equations is equivalent with the invariance of the
Poisson bracket and thus we can use this invariance as the definition for canonical
transformations.

% An other equivalent condition would require the Jacobian matrix
% \(\frac{\partial(Q,P)}{\partial(q,p)}\) to be {\color{red}symplectic}.

\section{The Hamilton-Jacobi equation}

This section will describe the Hamilton-Jacobi formalism. For further details,
one can see for example~\cite{Hand1998, Goldstein2002}.
Instead of finding the equations of motion through solving a set of \(2n\) ordinary differential
equations with \(2n\) dependent variables (\(q_1, \dotsc, q_n\) and \(p_1, \dotsc, p_n\))
and one independent variable (\(t\)), we can use a single first order partial differential equation
having \(n+1\) independent variables (\(q_1, \dotsc, q_n\) and \(t\)) and one
dependent variable (\(S\)).

Suppose there exists a generating function \(S\) of type \(F_2\) such that
\[
  \mathcal{H'} = 0.
\]

From Hamilton's equations, the new coordinates \(Q_k\) and \(P_k\), with \(k=1,\dotsc,n\) will
be constants, so the information about the time evolution of the system is contained
in the canonical transformation itself.Thus, equation~\eqref{eq:canonical-transf} becomes
\begin{equation}
  \label{eq:hamilton-jacobi-first}
  \mathcal{H}(q_1,\dotsc,q_n,\pdv{S}{q_1},\dotsc,\pdv{S}{q_n},t) + \pdv{S}{t} = 0
\end{equation}
which represents the \emph{Hamilton-Jacobi equation}.

The solution \(S(q_1,\dotsc,q_n,P_1,\dotsc,P_n,t)\) of
equation~\eqref{eq:hamilton-jacobi-first} is called \emph{Hamilton's Principal Function}.
From the theory of differential equations we know that there will be \(n+1\) arbitrary constants.
We can find \(n\) of these constants as \(P_1,\dotsc,P_n\) and the remaining constant can be
incorporated in \(S\) since if \(S\) is a solution for the Hamilton-Jacobi equation,
\(S+const.\) is also a solution.

It is a standard convention~\cite{Goldstein2002} to rename the constants as follows
\begin{align*}
  P_k &\equiv \alpha_k \\
  Q_k &\equiv \beta_k = \pdv{S(q_1,\dotsc,q_n,\alpha_1,\dotsc,\alpha_n,t)}{\alpha_k}.
\end{align*}
These constants depend on the initial conditions \(q_k(0)\) and \(p_k(0)\). From
\[
  p_k(0) = \eval{\pdv{S(q_1,\dotsc,q_n,\alpha_1,\dotsc,\alpha_n,t)}{q_k}}_{t=0}
\]
we can obtain \(\alpha_k\)s, and \(\beta_k\)s are given by
\[
  \beta_k = \eval{\pdv{S(q_1,\dotsc,q_n,\alpha_1,\dotsc,\alpha_n,t)}{\alpha_k}}_{t=0}.
\]

We can obtain the initial generalised coordinates
\(q_k(\alpha_1,\dotsc,\alpha_n,\beta_1,\dotsc,\beta_n,t)\) by solving
\[
  \beta_k=\pdv{S}{\alpha_k}.
\]

If we take the total time derivative of \(S\)
\[
  \dv{S}{t} = \sum_{k=0}^{n} \pdv{S}{q_k} \dot{q_k} + \pdv{S}{t}
\]
and use eq.~\eqref{eq:canonical-transf}
\[
  \dv{S}{t} = \sum_{k=0}^{n} p_k \dot{q_k} - \mathcal{H},
\]
we recognise the Legendre transform from eq.~\eqref{eq:legendre-tr}. Thus
\[
  \dv{S}{t} = \mathcal{L},
\]
or equivalently
\[
  S = \int \mathcal{L} \dd{t}.
\]
It follows that the generating function for this special canonical transformation
is the action as defined in eq.~\eqref{eq:action}.

If the time does not appear explicitly in the Hamiltonian, the Hamiltonian itself
will be a constant of the motion and will represent the energy of the system \(E\).

We can rewrite eq.~\eqref{eq:hamilton-jacobi-first} as follows
\[
  \mathcal{H}(q_1,\dotsc,q_n,\pdv{S}{q_1},\dotsc,\pdv{S}{q_n},t) = - \pdv{S}{t} = E.
\]
Thus we can suppose that for this case \(S(q_1,\dotsc,q_n,\alpha_1,\dotsc,\alpha_n,t)\)
has the following form
\[
  S(q_1,\dotsc,q_n,\alpha_1,\dotsc,\alpha_n,t) \equiv W(q_1,\dotsc,q_n,\alpha_1,\dotsc,\alpha_n) - Et.
\]
Using this expression in eq.~\eqref{eq:hamilton-jacobi-first}, we obtain
\begin{equation}
  \label{eq:hamilton-jacobi-second}
  \mathcal{H}(q_1,\dotsc,q_n,\pdv{W}{q_1},\dotsc,\pdv{W}{q_n}) = E.
\end{equation}
which represents the second form of the Hamilton-Jacobi equation.

The solution to this equation is called \emph{Hamilton's Characteristic Function}.
\(W\) can be expressed as
\[
  W = S + Et = \int \left(\mathcal{L} + \mathcal{H} \right) \dd{t}
    = \int \sum_{k=0}^n p_k \dot{q_k} \dd{t}
    = \sum_{k=0}^n \int p_k \dd{q_k}.
\]
% {\color{red}physical interpretation?}

\(W\) is a generating function of type \(F_2\). The canonical transformation given by \(W\)
is different from the one given by \(S\), since
\(\mathcal{H'} = \mathcal{H} + \pdv{W}{t} = E \neq 0\).

The new generalised coordinates \(Q_k\) are cyclic in the new Hamiltonian
\(\mathcal{H'} = E = \mathcal{H}(q_1,\dotsc,q_n,\pdv{W}{q_1},\dotsc,\pdv{W}{q_n})\),
thus the new generalised momenta care constants. While in the case of \(S\)
all the generalised momenta and coordinates were constants, in the case of \(W\)
all the generalised momenta are constants \(\dot{P_k}=-\pdv{\mathcal{H'}}{Q_k}=0\),
but not all the generalised coordinates are constants.
If we choose to define \(P_1 \equiv E\),
\[
  \dot{Q_1} = \pdv{\mathcal{H'}}{P_1} = \pdv{E}{P_1} = 1.
\]
Thus
\[
  Q_1 = t + const. = \pdv{W}{P_1} = \pdv{W}{E}.
\]
So \(Q_1\) is no longer a constant. The constant in the above equation determines the
origin of the time axis. \(Q_1\) represents the time and its conjugate coordinate is
\(P_1\), the energy.

The \(n-1\) generalised momenta \(P_2,\dotsc,P_n\) are independent of \(P_1\)
and thus \(\pdv{\mathcal{H'}}{P_k} = 0\), where \(k=2,\dotsc,n\). So the corresponding
\(n-1\) generalised coordinates are constants.

We can change the notation for \(\alpha \) and \(\beta \) to use an index from \(1\) to \(n-1\):
\(\alpha_j \equiv P_{j+1},\ \beta_j \equiv Q_{j+1}\), where \(j=1,\dotsc,n-1\).
We can now express Hamilton's Characteristic Function as
\[
  W=W(q_1,\dotsc,q_n,E,\alpha_1,\dotsc,\alpha_{n-1}).
\]

From the initial generalised momenta \(p_k(0)\)
\[
  p_k(0) = \eval{\pdv{W(q_1,\dotsc,q_n,E,\alpha_1,\dotsc,\alpha_{n-1})}{q_k}}_{t=0}
\]
we can obtain \(E\) and \(\alpha_j\). Using the initial generalised coordinates \(q_k(0)\)
we obtain \(\beta_j\) as follows
\[
  \beta_j = \eval{\pdv{W(q_1,\dotsc,q_n,E,\alpha_1,\dotsc,\alpha_{n-1})}{\alpha_j}}_{t=0}.
\]
Thus, having \(E, \alpha_j\) and \(\beta_j\), we can obtain the time evolution
for the generalised coordinates from
\[
  \beta_j = \pdv{W(q_1,\dotsc,q_n,E,\alpha_1,\dotsc,\alpha_{n-1})}{\alpha_j}.
\]

For systems with \(n>1\) degrees of freedom, the Hamilton-Jacobi equation can be separated in
\(n\) equations if we can write \(S\) as
\[
  S(q_1,\dotsc,q_n,\alpha_1,\dotsc,\alpha_n,t) = \sum_{k=1}^n W_k(q_k,\alpha_1,\dotsc,\alpha_n)-Et.
\]
A simpler form of separability exists if the Hamiltonian itself is separable
\[
  \mathcal{H}(q_1,\dotsc,q_n,p_1,\dotsc,p_n) = \sum_{k=1}^n \mathcal{H}_k(q_k,p_k).
\]
In this case the second form of the Hamilton-Jacobi equation splits in \(n\) independent equations
\[
  \mathcal{H}(q_1,\dotsc,q_n,p_1,\dotsc,p_n) = E = \sum_{k=1}^n \mathcal{H}_k(q_k,p_k),
\]
so
\[
  \mathcal{H}_k(q_k,\pdv{W_k}{q_k}) = \alpha_k,\qquad E = \sum_{k=1}^n \alpha_k.
\]
% {\color{red}why \(E = \sum_{k=1}^n \alpha_k\)?}

The Hamilton-Jacobi equation can be solved only if the variables can be separated.
The separability depends both on the problem and the chosen coordinate system.
% For orthogonal coordinates the \emph{Staeckel conditions}
% (see Goldstein pp. 453) establish which kind of potentials can be separated.
This does not decrease the theoretical value of the Hamilton-Jacobi equation which provides
powerful methods for finding the constants of the motion.

\section{Action-angle variables}

Going back again to a system with one degree of freedom for simplicity, we study a system
which has a periodic motion. We search for a canonical transformation from \(q,p\) to
a new set of canonically conjugate coordinates \(\theta, I\) such that the new
Hamiltonian does not depend on \(\theta \).

Since \(\theta \) is an cyclic coordinate
\[
  \dot{I} = -\pdv{\mathcal{H'}}{\theta} = 0,
\]
\(I\) is a constant of the motion and
\[
  \dot{\theta} = \pdv{\mathcal{H'}}{I} = const.
\]
Thus \(\theta = \pdv{\mathcal{H'}}{I} (t-t_0) \equiv \omega(I) (t-t_0)\).
\(\theta \) is called the \emph{angle variable} and \(I\) the \emph{action variable}
(not to be confused with the action).

In order to obtain this variables we will use a generating function of type \(F_1\)
denoted \(W'(q,\theta)\). Since the motion is periodic in the initial coordinates, it
must also be periodic in the angle variable. Thus \(W'\) is periodic in \(\theta \).

From the properties of the \(F_1\) type generating functions
\[
  \dd{W'} = \pdv{W'}{q} \dd{q} + \pdv{W'}{\theta} \dd{\theta} = p \dd{q} - I \dd{\theta}.
\]
(we can observe that \(W'\) is related to Hamilton's Characteristic Function by a
Legendre transformation)

Since the motion is periodic, if we integrate over a period \(q\) returns to the same
value, while \(\theta \) advances by a constant amount, which we can choose to be
\(2\pi \) per period.
\[
  \oint \dd{W'} = 0 = \oint p \dd{q} - \oint I \dd{\theta}
\]
Since \(I\) is a constant of the motion,
\[
  2\pi I = \oint p \dd{q}
\]
or
\begin{equation}
  \label{eq:action-variable}
  I = \frac{1}{2\pi} \oint p \dd{q}.
\end{equation}
Equation~\eqref{eq:action-variable} can be considered the definition for the
action variable, where the integral is taken around a single period of the motion
(which is always a closed curve in phase space).

\section{Integrability}

A system is called \emph{integrable} if the number of degrees of freedom is equal to the number
of constants of the motion in involution which each other. Two constants of the motion
are in \emph{involution} with each other if their mutual Poisson brackets are equal to \(0\).

The previously mentioned constants of the motion are not required to be known in analytic form,
they are only required to exist. Thus a Hamiltonian system with one degree of
freedom, if the Hamiltonian function is time independent, is integrable.
The conserved quantity is the energy.

An important consequence of integrability is that we can always find a canonical
transformation to action-angle variables and express the new Hamiltonian as a function
of the \(n\) constants of the motion given by the action variables:
\(\mathcal{H'}=\mathcal{H'}(I_1,\dotsc,I_n)\).

For integrable systems, the \(n\) constants of the motion restrict the motion in the
\(2n\) dimensional phase space to a \(2n-n=n\) dimensional subspace. If the motion is
also periodic, then the motion is constrained to a \(n\) dimensional closed surface
(with finite volume).

For an integrable system with \(n=2\), the motion in the phase space is confined
to a torus. In order to understand why, we can use the previously defined action-angle variables.
In this case, the motion will be described by \(I_1, \theta_1, I_2, \theta_2\) and
the trajectory is constrained to a \(2\) dimensional subspace, obtained as a direct
product of two circles, having as radii the values of the two conserved quantities,
\(I_1\) and \(I_2\), while \(\theta_1\) and \(\theta_2\) will define the polar angles
of the two circular motions. Thus, the trajectory will be confined on the surface
of a torus.

To extend the discussion to systems with \(n\) degrees of freedom,
we replace the 2D plane with an \(n\) dimensional cube. If we join the sides of
this cube we obtain a \(n\) dimensional torus.

% {\color{red}Connection with the ``hairy ball'' theorem
%
% we consider a vector field associated with the tangent in each point of the trajectory
% in phase space. This vector field is generated by Hamilton's equations with time
% as parameter{\color{red} (why?/how?)}.
% In the \(n=2\) case we can see that if the the shape of the closed bi-dimensional
% surface were a sphere, we could not ``comb'' the vectors without creating a singularity.
% Instead, if the surface is a torus, the problem disappears. This result is a consequence of
% the ``hairy ball'' theorem. This theorem states that there is no non-vanishing continuous
% tangent vector field on even-dimensional $n$-spheres. (what happens for odd $n$?)
% }

% {\color{red}+ explanation for the fact that the constants of motion must be
% in involution with each other
%
% Observation:
% If we make a canonical transformation such that \(\mathcal{H'} = 0\), the new
% coordinates and momenta will be constants, so we have \(2n\) constants of the motion.
% If we make a canonical transformation to action-angle variables, we get
% \(n\) constants of the motion (the action variables).
% There is no contradiction since the \(2n\) constants obtained in the first case
% are not all in involution with each other (\( \{q_i,p_j\}=\delta_{i,j} \)), but
% we can choose \(n\) of them in involution with each other.
% }


\section{Chaos notions}

Deterministic chaos is a feature of non-integrable systems and is characterised
by high sensitivity to initial conditions: for two dynamical systems with
nearly identical initial conditions, the long time evolution is uncorrelated,
the motion in the two systems being very different.
Poincaré was the first to bring attention to this type of behaviour when he
proved that there are no analytic solutions to the 3-body problem~\cite{Diacu1996}.
Even if the equations of motion are deterministic, long time predictions are
rendered useless since an arbitrary small variation in the initial conditions
can produce very different results.

In chaotic systems, the motion is no longer constrained on tori as for the case
of integrable systems. The trajectories escape from the \(n\) dimensional subspace
in the \(2n\) phase space space, or in a \(2n-1\) subspace if the energy is
conserved.

For the following concepts we use mainly the notations in~\cite{Strogatz1994}.


\subsection{The Liapunov exponent}

Chaotic systems exhibit a varying degree of randomness. In order to describe
chaos quantitatively we introduce the \emph{Liapunov exponent}, which is a local measure
of the dispersion in phase space.
The sensitivity to the initial conditions of chaotic systems can be expressed
as an exponential divergence of neighbouring trajectories.

If two orbits are separated at \(t=0\) by the infinitesimal distance \(\dd{x_0}\),
then at a later time \(t\), the distance will be
\[
  \dd{x}(t) \sim \dd{x}(0) \ee^{\lambda t},
\]
where \(\lambda \) is the Liapunov exponent.
If \(\lambda > 1\), the motion is chaotic and \(\lambda \) gives the exponential
growth of the initial infinitesimal separation. If the motion is bounded,
\(\dd{x}(t)\) will grow until is comparable with the dimension of the
allowed part of the phase space, from that point on varying only randomly in time.
Thus the Liapunov exponent can be defined as follows
\begin{equation}
  \label{eq:liapunov}
  \lambda = \lim_{t \to \infty} \frac{1}{t} \ln{\frac{\dd{x}(t)}{\dd{x}(0)}}
\end{equation}

For systems with discrete time evolution instead of the time we can use the number
of iterations.

\subsection{Poincaré sections}

Since the phase space is \(2n\) dimensional, visualising the motion can be
difficult even for systems with \(n=2\) degrees of freedom. If the energy is
constant, the motion will be constrained in the \(2n-1\) subspace, but even for
the case of \(n=2\) the trajectory is not easy to follow in a 3D space considering
that in most cases is represented on a 2D surface.

For \(n=2\) systems this difficulty can be addressed by using a slice of the phase
space which reduces the dimensionality by $1$. This slices are called
\emph{Poincaré sections}.
If the energy is conserved, the section will be a plane.

If \(\Sigma \) is a Poincaré section intersecting the trajectories of the system
in phase space in the points \(\vb{x}_i\), then we can define the \emph{Poincaré map}
as a mapping \({P:\Sigma \to \Sigma}\) such that
\[
  \vb{x}_{i+1} = P(\vb{x}_i).
\]

A point \(\vb{x}^*\) is called a \emph{fixed point} of the  if
\[
  \vb{x}^* = P(\vb{x}^*).
\]
Thus a fixed point belongs to a closed trajectory. In order to study the stability
of the orbit, we can analyse the stability of the fixed point.
If \(\vb{x}^*\) is a fixed point, we consider a small perturbation in the initial
conditions and we look at what happens to a point \(\vb{x}^* + \vb{v}_i\) near the fixed point
\[
  P(\vb{x}^* + \vb{v}_i) = P(\vb{x}^*) + \eval{\pdv{P}{\vb{x}}}_{\vb{x}=\vb{x}^*} \vb{v}_i + \order{\norm{\vb{v}_i}^2},
\]
where \(\pdv{P}{\vb{x}}\) is the Jacobian of the map.
Assuming we can neglect \(\order{\norm{\vb{v}_i}^2}\), we get
\[
  \vb{x}^* + \vb{v}_{i+1} = \vb{x}^* + \eval{\pdv{P}{\vb{x}}}_{\vb{x}=\vb{x}^*} \vb{v}_i,
\]
where \(\vb{x}^* + \vb{v}_{i+1} = P(\vb{x}^* + \vb{v}_i)\).%{\color{red} (\(P(\vb{x}^* + \vb{v}_i) = P(\vb{x}^*) + P(\vb{v}_i)\)?)}
Thus
\begin{equation}
  \label{eq:lin-map}
  \vb{v}_{i+1} = \eval{\pdv{P}{\vb{x}}}_{\vb{x}=\vb{x}^*} \vb{v}_i,
\end{equation}
also called the \emph{linearised Poincaré map}.

In a basis given by the eigenvectors \( \{\vb{e}_j\} \) of the Jacobian,
\(\vb{v}_{i} = \sum_{j=1}^{2n-1} a_j \vb{e}_j\), then
\[
  \vb{v}_{i+1} = \eval{\pdv{P}{\vb{x}}}_{\vb{x}=\vb{x}^*} \sum_{j=1}^{2n-1} a_j \vb{e}_j
               = \sum_{j=1}^{2n-1} \eval{\pdv{P}{\vb{x}}}_{\vb{x}=\vb{x}^*} \vb{e}_j a_j
               = \sum_{j=1}^{2n-1} \lambda_j a_j \vb{e}_j,
\]
where \(\lambda_j\) are the eigenvalues of the Jacobian matrix.

If we consider \(k\) iterations of eq.~\eqref{eq:lin-map}
\[
  \vb{v}_{i+k} = \sum_{j=1}^{2n-1} {(\lambda_j)}^k a_j \vb{e}_j
\]
If all \(\abs{\lambda_j} < 1\), then \(\norm{\vb{v}_{i+k}} \to 0\) as \(k\to\infty \)
and the fixed point is \emph{linearly stable}. If \(\abs{\lambda_j} > 1\) for
some \(j\), then the fixed point is \emph{linearly unstable}. In the case when
\(\abs{\lambda_m}=1\), where \(\lambda_m\) is the largest eigenvalue, we cannot
decide and a nonlinear stability analysis is required.


\section{Kolmogorov-Arnold-Moser theorem}

Since most systems are non-integrable we need approximate methods. One such
method is canonical perturbation theory which can provide approximate solutions
when the non-integrable Hamiltonian can be broken in an integrable part and
a relatively small perturbation. For a detailed treatment, see~\cite{Goldstein2002}.

Let us consider
\[
  \mathcal{H} = \mathcal{H}_0 + \Delta \mathcal{H},
\]
where \(\mathcal{H}_0\) is the integrable part of the Hamiltonian for
which the solutions of the Hamilton equations are known and
\(\Delta \mathcal{H}\) is the perturbation.

If we make a canonical transformation with a generating function of type
\(F_2\) such that \(\mathcal{H'}_0 = 0\) we can obtain the constants of the
motion for the integrable part of the Hamiltonian, \(Q_0\) and \(P_0\).
If we use the generating function \(S(q,P,t)\) to make ca canonical transformation
for \(\mathcal{H}\), we obtain
\[
  \mathcal{H'}(Q_0,P_0,t) = \Delta \mathcal{H}^0(Q_0,P_0,t).
\]
Hamilton's equations read
\begin{align*}
  \dot{Q} = \pdv{\mathcal{H'}}{P} = \pdv{\Delta \mathcal{H}^0}{P}, &&
  \dot{P} = -\pdv{\mathcal{H'}}{Q} = -\pdv{\Delta \mathcal{H}^0}{Q}.
\end{align*}

In order to obtain the first order correction, we replace \(Q\) and \(P\)
after derivation with \(Q_0\) and \(P_0\).
\begin{align*}
  \dot{Q_1} = \eval{\pdv{\Delta \mathcal{H}^0}{P}}_{\substack{Q=Q_0,\\P=P_0}} &&
  \dot{P_1} = \eval{-\pdv{\Delta \mathcal{H}^0}{Q}}_{\substack{Q=Q_0,\\P=P_0}}
\end{align*}
We can integrate the above differential equations to obtain \(Q_1\) and \(P_1\).

% {\color{red}With these corrections we can construct a new generating function
% \(S(Q_1,P_1,t)\) (why not \(S(q,P_1,t)\)?
% will this generating function be of type \(F_2\)?) which we can use to obtain
% a new perturbed Hamiltonian \(\Delta \mathcal{H}^1\) (is \(\mathcal{H''}_0 = 0\)?)}.

In order to obtain higher order corrections we replace in \(\Delta \mathcal{H}^0\)
the conjugate variables \(Q_0\) and \(P_0\) with \(Q_1\) and \(P_1\),
obtaining \(\Delta \mathcal{H}^1\) and the procedure repeats.
\begin{align*}
  \dot{Q}_{i+1} = \eval{\pdv{\Delta \mathcal{H}^i}{P}}_{\substack{Q=Q_i,\\P=P_i}} &&
  \dot{P}_{i+1} = \eval{-\pdv{\Delta \mathcal{H}^i}{Q}}_{\substack{Q=Q_i,\\P=P_i}}
\end{align*}

It is clear that if the perturbation is small enough, we can no longer apply
this method and the perturbation will significantly alter the motion.
Hence, an important question that arises is the stability of the perturbed solutions
and the domain of applicability for the perturbation theory.

The Kolmogorov-Arnold-Moser theorem or K.A.M. theorem gives the limits of regular (or
non-chaotic) motion for systems with bounded motion.


\begin{theorem*}[Kolmogorov-Arnold-Moser theorem]
  If the bounded motion of an integrable system given by the Hamiltonian \(\mathcal{H}_0\)
  is disturbed by a small perturbation \(\Delta \mathcal{H}\) rendering the system
  non-integrable and if the unperturbed motion has incommensurate
  frequencies \(\omega_i=\dot{\theta}_i\), where \(\theta_i\) are the angle
  variables, then the motion will remain confined to an \(n\)-torus, except for
  a small set of initial conditions for which the trajectories escape on the
  energy hyper-surface.
\end{theorem*}

The theorem tells us that the majority of the tori will survive the perturbation
and the set of destroyed tori has finite volume in the phase space.
A torus of the unperturbed system survives if it can be continuously deformed
in a torus of the perturbed system.
Furthermore, the destroyed tori have rationally related frequencies, that is
there exist \(m_1,\dotsc,m_n \equiv \vb{m} \in \mathbb{Z}^n\) such that
\(\vb{m} \vdot \vb*{\omega} = 0\), where \(\vb*{\omega} \equiv \omega_1,\dotsc,\omega_n\).

Since the set of rational numbers is dense in the set of real numbers,
the set of destroyed tori will be dense in the phase space. In the
place of the destroyed tori chaotic motion will appear as the trajectories escape
in the \(2n-1\) subspace of constant energy.


\end{document}  % chktex 17
