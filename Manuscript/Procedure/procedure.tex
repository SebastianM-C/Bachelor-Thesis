\documentclass[../thesis.tex]{subfiles}

%!TeX spellcheck = en-GB

% chktex-file 18

\begin{document}

\chapter{Procedure}


\section{The Hamiltonian}

In this chapter we will describe how we obtained the numerical results.
We begin by computing the matrix elements of the Hamiltonian in a basis
given by the eigenstates of an isotropic double harmonic oscillator.
In analogy with the classical case, we consider two independent
quantum numbers \(n_1, n_2\) corresponding to the two orthogonal
oscillating directions. We can define two number operators \(N_1, N_2\) such that
\(N_1 \ket{n_1, n_2} = n_1 \ket{n_1, n_2}\) and \(N_2 \ket{n_1, n_2} = n_2 \ket{n_1, n_2}\).
We consider the creation and annihilation operators \(\ad{1}, a_1\) and
\(\ad{2}, a_2\)
such that \(  N_1 = \ad{1} a_1 \) and \( N_2 = \ad{2} a_2 \).
In terms of the previously defined operators, the Hamiltonian of the isotropic
double harmonic oscillator is given by
\[
  H_0 = \hbar \omega_0 \left(\ad{1} a_1 + \frac{1}{2} I + \ad{2} a_2 + \frac{1}{2} I\right)
      = \hbar \omega_0 \left(N_1 + N_2 + I\right)
\]
and its eigenstates are given by
\[
  H_0 \ket{n_1, n_2} = \hbar \omega_0 \left( n_1 + n_2 + 1 \right) \ket{n_1, n_2}
\]

Since \(n_1 + n_2 = n\) can be obtained in \(\sum_{i=0}^n i = \frac{1}{2}\,n(n+1)\) ways,
the energy levels of the isotropic double harmonic oscillator are \(\frac{1}{2}\,n(n+1)\) fold
degenerated.

In the fundamental state \(\ket{0, 0}\), with \(n_1=n_2=0\), the oscillator has the
energy equal to \(\hbar \omega_0\),
having a \emph{zero point motion} {\color{red} (details / definition?)}.
Since we are interested {\color{red} (why?)} in variations from the equilibrium state,
we can rescale the potential energy such that the energy of the fundamental state becomes \(0\).
Thus the new Hamiltonian will be given by
\[
  H_0 = A \left(N_1 + N_2\right),
\]
where \(A = \hbar \omega_0\).

We can construct the basis starting from the vacuum state \(\ket{0, 0}\) by acting
with the creation operators
\begin{align*}
  {\ad{1}}^{n_1} \ket{0, 0} &= \sqrt{n_1!} \ket{n_1, 0} \\
  {\ad{2}}^{n_2} \ket{0, 0} &= \sqrt{n_2!} \ket{0, n_2}
\end{align*}

Thus, by applying the operator \({\ad{1}}^{n_1} {\ad{2}}^{n_2}\)
such that \(n_1 + n_2 = n\) (in order to obtain for each \(n\) all the states with the
energy \(n\, \hbar \omega_0\) {\color{red} wrt vacuum}) we obtain the basis elements ordered as follows:
\[
  \ket{0, 0} \ket{0, 1} \ket{0, 2} \cdots \ket{0, n} \ket{1, 0} \ket{1, 1}
  \cdots \ket{1, n - 1} \cdots \ket{i, 0} \ket{i, 1} \cdots \ket{i, n - i} \cdots
  \ket{n, 0}.
\]

For our investigations, the Hamiltonian is expressed as a function of the creation and annihilation
operators up to fourth order terms as follows

\begin{equation}
\label{eq:hamilt}
\begin{split}
  H &= A \left( \ad{1} a_1 + \ad{2} a_2 \right)
    + \frac{B}{4} \bigg[ \left( 3 \ad{1} {\ad{2}}^2 + 3 a_1 a_2^2
                               - {\ad{1}}^3 - a_1^3 \right)   \\
  &\quad + 3 \left( a_1 {\ad{2}}^2 + \ad{1} a_2^2 - \ad{1} a_1^2 - {\ad{1}}^2 a_1
             + 2 a_1 \ad{2} a_2 + 2 \ad{1} \ad{2} a_2
          \right) \bigg]  \\
  &\quad + \frac{D}{16} \bigg[ 6 \left( {\ad{1}}^2 a_1^2 + {\ad{2}}^2 a_2^2 \right)
                        + 2 \left( a_1^2 {\ad{2}}^2 + {\ad{1}}^2 a_2^2 \right)
                        + 8 \ad{1} a_1 \ad{2} a_2  \\
  &\quad + 4 \left(\ad{1} a_1^3 + {\ad{1}}^3 a_1 + \ad{2} a_2^3 + {\ad{2}}^3 a_2
     + a_1^2 \ad{2} a_2 + {\ad{1}}^2 \ad{2} a_2 + \ad{1} a_1 a_2^2 + \ad{1} a_1 {\ad{2}}^2
        \right)  \\
  &\quad + \left( {\ad{1}}^4 + a_1^4 + {\ad{2}}^4 + a_2^4
     + 2 {\ad{1}}^2 {\ad{2}}^2 + 2 a_1^2 a_2^2
      \right)
                        \bigg].
\end{split}
\end{equation}

The physical origin of this Hamiltonian is related to quadrupole vibrations of
nuclear surfaces. {\color{red} (details for B, D?) (add citations)}
The energy levels will be expressed in units of harmonic oscillator energy and therefore
from here on we will consider \(A = 1\).
We can obtain the eigenvalues and eigenvectors of the Hamiltonian by a
diagonalization routine based on Relatively Robust Representations from
Intel\textsuperscript{\textregistered} Math Kernel Library used via a \texttt{Python}
program{\color{red} (+ citations)}.
Any such diagonalization method requires a truncation of the Hilbert space
which induces errors concerning the eigenvalues. This errors increase as one moves to
the upper limit of the energy for a fixed dimension of the Hilbert space. Indeed we
expect this energies to have more important contributions from the states that were
eliminated by truncation.
We tested the stability of the energy levels by comparing the results obtained for
different sizes of the diagonalization basis as is detailed in the next section.

\section{Stability}

We consider the \emph{stable levels} to be the eigenvalues which, at a change of
basis from one with a dimension of $N$ to one with dimension \(N+ \Delta N\),
do not change with more than a chosen threshold \(\delta_s\).
In the following figure we show the variation of the energy levels when
the dimension increases from \(N = 120\) to \(N = 140\).

\begin{center}
  \includegraphics{"B0.2 D0.4 N120/bar_E_diff"}
  \captionof{figure}{\(B = 0.2, D = 0.4, N = 120\)}
\end{center}

We can observe that the first 400--600 eigenvalues have a very good stability.
Thus we can choose the stability threshold for example at \(\delta_s = 10^{-9}\).
Qualitatively the shape of this distribution does not depend on the parameters of
the Hamiltonian or the dimension of the Hilbert space because it reflects the nature
of the approximation as discussed previously.
For example, for \(B = 0.55, D = 0.4, N = 260\) compared
with \(N = 280\)

\begin{center}
  \includegraphics{"B0.55 D0.4 N260/bar_E_diff"}
  \captionof{figure}{\(B = 0.55, D = 0.4, N = 260\)}
\end{center}

As expected, the number of stable levels increases with the size of the basis.
Roughly, for a given basis size $N$, the first 7--8\% levels differ with less
than \(\delta_s = 10^{-9}\) when we compare with a basis of dimension
\(N + \Delta N\), with \(\Delta N = 20\).

\section{Statistics}

As we mentioned in the Introduction, a spectrum can be characterised through the
probability distribution of the nearest neighbour spacing.
The \emph{spacing} is defined as the difference between two consecutive
energy levels. Similarly the \emph{relative spacing} is defined as
\[
  s = \frac{E_{i+1} - E_i}{\mean{\Delta E}},
\]
where \(\mean{\Delta E}\) is the average spacing \(\frac{E_n - E_0}{N}\).

The \emph{nearest neighbour spacing distributions} tell us the probability \(P(s)\dd{s}\) to
find a relative spacing $s$ when we move in the spectrum obtained by diagonalization.
This probability is defined as follows
\[
  P(s)\dd{s} = \frac{N_{s,s+\Delta s}}{N},
\]
where \(N_{s,s+\Delta s}\) is the number of levels with the relative spacing
between \(s\) and \(s+ \Delta s\).
We can also define a \emph{cumulative probability distribution},
\[
  I(s) = \sum_{s_i=0}^s P(s_i) \Delta s
\]

\subsection{Irreducible representations}

In order to analyse the fluctuations of the previously obtained eigenvalues
we must first take into account the symmetry of the system. The symmetries of
a system can make influence the fluctuations making the given sequence of levels
appear more regular than it actually is {\color{red} (better explanation \& citation)}.
In order to remove this influence we analyse separately each symmetry reduced subspace
or equivalently each \emph{irreducible representation} of the Hamiltonian.

If an operator $A$ corresponds to a symmetry of the Hamiltonian, then it
commutes with the Hamiltonian, \(\comm{A}{H} = 0\). {\color{red} (Proof?)}
For the Hamiltonian in eq.~\eqref{eq:hamilt} we have the following symmetries {\color{red} (?)},
which correspond to the symmetry group \(\mathcal{C}_{3v}\).
This group has 3 irreducible representations: one bi-dimensional and
two unidimensional one symmetric and one anti-symmetric, namely
\(\Gamma_b, \Gamma_s, \Gamma_a\){\color{red} (notation?)}.

\subsubsection{Separating the bi-dimensional representation}

The presence of the bi-dimensional representation corresponds a two-fold
degeneracy which allows us to identify it by computing the
differences between consecutive levels \(\Delta E = E_{i+1} - E_i\).
An other option is to use directly the relative spacing, which differs only by a
constant from \(\Delta E\), namely the average spacing.
The separation of the symmetric and anti-symmetric irreducible representations
will be detailed later.

In figure~\ref{fig:bar_delta} we can see how \(\Delta E\) varies with the index of the
levels.

\begin{figure}[!h]
  \centering
  \includegraphics{"B0.2 D0.4 N120/bar_delta"}   % TODO: change with the correct one
  \caption{\(B=0.2, D=0.4, N=120\)}%
\label{fig:bar_delta}
\end{figure}

Because of the finite precision of the numerical implementation, the
difference between two consecutive degenerate levels might not be exactly 0,
its value depending on the machine precision (as it can be seen in the above figure).
To take this fact into account we will consider that the levels which
have $s$ (or \(\Delta E\)) greater than a chosen \( \varepsilon \) belong
to one of the unidimensional representations.
In order to choose a suitable value for \( \varepsilon \), we use a histogram
to visualise the number of levels at different spacings
(see figure~\ref{fig:relsp-b0.2n120}).

\begin{figure}[h!]
  \centering
  \begin{subfigure}[b]{0.49\textwidth}
    \centering
    \includegraphics{"B0.2 D0.4 N120/hist_relsp"}
    \caption{\(B=0.2, D=0.4, N=120\)}%
    \label{fig:relsp-b0.2n120}  % chktex 24
  \end{subfigure}
  \begin{subfigure}[b]{0.49\textwidth}
    \centering
    \includegraphics{"B0.4 D0.4 N260/hist_relsp"} % TODO: change with the correct one
    \caption{\(B=0.4, D=0.4, N=260\)}%
    \label{fig:relsp-b0.4n260}  % chktex 24
  \end{subfigure}
  \caption{The relative spacing histograms for different parameters}
\end{figure}

This bimodal shape of the histogram suggests clearly the presence of the degenerate
levels well separated from the rest.
For some particular values for $B$ (such as \(B = 0.4\)) and high values for
$N$ (\( N > 200 \)), we observed a splitting of the block corresponding to
the degenerated levels in two blocks, namely one at exactly 0 and the other at low values.
(see fig.~\ref{fig:relsp-b0.4n260})

If we plot the spacing as a function of the level index, we can see how
each level is situated with respect to the chosen \( \varepsilon \).
Once again we can observe how the spacings corresponding to the bi-dimensional
representation are separated from the rest.

\begin{figure}[H]
  \centering
  \includegraphics{"B0.2 D0.4 N120/relsp"}
  \caption{\(B=0.2, D=0.4, N=120\)}
\end{figure}

\subsubsection{Separating the unidimensional representations}

In order to identify the unidimensional representations we rely on the symmetry
of the potential energy operator,
\[
  \color{red}\dots ?
\]
We can observe that this operator is invariant to a reflection along the
{\color{red}\(Ox\)?} axis.
{\color{red} axis \(\rightarrow n_2\)?}
For the harmonic oscillator the states with even quantum numbers have even wave functions.
{\color{red} (proof / cite?)}
Thus, we can establish a correspondence {\color{red} (how?)} between the symmetry of the
representation and the parity of the quantum number \(n_2\) such that the
unidimensional symmetric representation corresponds to an even \(n_2\) and the
unidimensional anti-symmetric representation corresponds to an odd \(n_2\).

The values of the quantum numbers \(n_1\) and \(n_2\) depend on the ordering of the basis.
Because the diagonalisation algorithm returns the eigenvalues
(and the corresponding eigenvectors) in ascending order,
the initial ordering of the basis is lost.
One method to approximate \(n_1\) and \(n_2\) for a given eigenvector would be
to consider that their values are given by the index of the dominant coefficient.
Thus, if we have the following eigenvector
\[
\begin{pmatrix}
    C_{0,0} \\ C_{0,1} \\ \vdots \\ C_{0,n} \\ C_{1,0} \\ \vdots \\ C_{1,n-1} \\
    \vdots \\ C_{i,0} \\ \vdots \\ C_{i,n-i} \\ \vdots \\ C_{n,0}
  \end{pmatrix}
\]
and \(C_{i,j}\) is the greatest coefficient, than we assign to this eigenvector
the quantum numbers of the \(k\)-th element in the basis, where \(k\) is the
index of the coefficient.

For example, for the simplified case of the isotropic double harmonic oscillator \
(\(B=D=0\)) with \(N=3\), the Hamiltonian is given by
\[
H=
\begin{pmatrix}
  0 & 0 & 0 & 0 & 0 & 0\\
  0 & 1 & 0 & 0 & 0 & 0\\
  0 & 0 & 2 & 0 & 0 & 0\\
  0 & 0 & 0 & 1 & 0 & 0\\
  0 & 0 & 0 & 0 & 2 & 0\\
  0 & 0 & 0 & 0 & 0 & 2\\
\end{pmatrix}
\]
The eigenvalues will be \(E_i = 0, 1, 1, 2, 2, 2\) with the corresponding
eigenvectors
\[
  v_1 =
  \begin{pmatrix}
    1 \\ 0 \\ 0 \\ 0 \\ 0 \\ 0
  \end{pmatrix},\
  v_{2} =
  \begin{pmatrix}
    0 \\ 1 \\ 0 \\ 0 \\ 0 \\ 0
  \end{pmatrix},\
  v_{3} =
  \begin{pmatrix}
    0 \\ 0 \\ 0 \\ 1 \\ 0 \\ 0
  \end{pmatrix},\
  v_{4} =
  \begin{pmatrix}
    0 \\ 0 \\ 1 \\ 0 \\ 0 \\ 0
  \end{pmatrix},\
  v_{5} =
  \begin{pmatrix}
    0 \\ 0 \\ 0 \\ 0 \\ 0 \\ 1
  \end{pmatrix},\
  v_{6} =
  \begin{pmatrix}
    0 \\ 0 \\ 0 \\ 0 \\ 1 \\ 0
  \end{pmatrix}
\]

Since the basis is given by
\[
  \ket{0, 0} \ket{0, 1} \ket{0, 2} \ket{1, 0} \ket{1, 1} \ket{2, 0},
\]
the quantum numbers for the eigenvectors will be assigned as follows
\begin{align*}
  v_1 &\equiv \ket{0, 0} \text{since}\ k = 1 \\
  v_2 &\equiv \ket{0, 1} \text{since}\ k = 2 \\
  v_3 &\equiv \ket{1, 0} \text{since}\ k = 4 \\
  v_4 &\equiv \ket{0, 2} \text{since}\ k = 3 \\
  v_5 &\equiv \ket{2, 0} \text{since}\ k = 6 \\
  v_6 &\equiv \ket{1, 1} \text{since}\ k = 5
\end{align*}

{\color{red} Maximum is not unique? \\ \centerline{\dots}}

In figures~\ref{fig:bar-rep-b0.2n120} and~\ref{fig:hist-rep-b0.2n120}
we can see the relative spacing for each irreducible representation as a function
of index and as a histogram.

% B=0.2 N=120
\begin{figure}
  \centering
  \begin{subfigure}[b]{0.49\textwidth}
    \centering
    \includegraphics{"B0.2 D0.4 N260/bar_rebde"}
  \end{subfigure}
  \begin{subfigure}[b]{0.49\textwidth}
    \centering
    \includegraphics{"B0.2 D0.4 N260/bar_reuna"}
  \end{subfigure}
  \begin{subfigure}[b]{0.49\textwidth}
    \centering
    \includegraphics{"B0.2 D0.4 N260/bar_reuns"}
  \end{subfigure}
  \caption{The relative spacing for each irreducible representation as a function
  of index for \(B=0.2, D=0.4, N=260\)}
  \label{fig:bar-rep-b0.2n120}  % chktex 24
\end{figure}

\begin{figure}
  \centering
  \begin{subfigure}[b]{0.49\textwidth}
    \centering
    \includegraphics{"B0.2 D0.4 N260/rebde"}
  \end{subfigure}
  \begin{subfigure}[b]{0.49\textwidth}
    \centering
    \includegraphics{"B0.2 D0.4 N260/reuna"}
  \end{subfigure}
  \begin{subfigure}[b]{0.49\textwidth}
    \centering
    \includegraphics{"B0.2 D0.4 N260/reuns"}
  \end{subfigure}
  \caption{The relative spacing histogram for each irreducible representation for
  \(B=0.2, D=0.4, N=260\)}
\label{fig:hist-rep-b0.2n120}
\end{figure}

% B=0.63 N=260
\begin{figure}
  \centering
  \begin{subfigure}[b]{0.49\textwidth}
    \centering
    \includegraphics{"B0.63 D0.4 N260/bar_rebde"}
  \end{subfigure}
  \begin{subfigure}[b]{0.49\textwidth}
    \centering
    \includegraphics{"B0.63 D0.4 N260/bar_reuna"}
  \end{subfigure}
  \begin{subfigure}[b]{0.49\textwidth}
    \centering
    \includegraphics{"B0.63 D0.4 N260/bar_reuns"}
  \end{subfigure}
  \caption{The relative spacing for each irreducible representation as a function
  of index for \(B=0.63, D=0.4, N=260\)}
  \label{fig:bar-rep-b0.63n260}  % chktex 24
\end{figure}

\begin{figure}
  \centering
  \begin{subfigure}[b]{0.49\textwidth}
    \centering
    \includegraphics{"B0.63 D0.4 N260/rebde"}
  \end{subfigure}
  \begin{subfigure}[b]{0.49\textwidth}
    \centering
    \includegraphics{"B0.63 D0.4 N260/reuna"}
  \end{subfigure}
  \begin{subfigure}[b]{0.49\textwidth}
    \centering
    \includegraphics{"B0.63 D0.4 N260/reuns"}
  \end{subfigure}
  \caption{The relative spacing histogram for each irreducible representation for
  \(B=0.63, D=0.4, N=260\)}
\label{fig:hist-rep-b0.63n260}
\end{figure}

We can plot the average spacing as a function of $B$ (see fig.~\ref{fig:avgsp})
{\color{red} (details?)}

\begin{figure}
  \centering
  \includegraphics{"avg_sp_N[220, 240, 260]"}
  \caption{\(\mean{\Delta E}\) as a function of $B$ for each representation}
\label{fig:avgsp}
\end{figure}

{\color{red} Selection problems?}



\begin{figure}
  \includegraphics{"B0.2 D0.4 N120/P(s)_st_1e-09_eps_1e-08"}  % chktex 36
  \caption{\(B=0.2, D=0.4, N=120\)}
\label{fig:P(s)-b0.2n120}
\end{figure}

\begin{figure}
  \includegraphics{"B0.63 D0.4 N260/P(s)_st_1e-09_eps_1e-08"}  % chktex 36
  \caption{\(B=0.63, D=0.4, N=260\)}
\label{fig:P(s)-b0.63n260}
\end{figure}

\begin{figure}
  \includegraphics{"B0.2 D0.4 N120/I(s)"}  % chktex 36
  \caption{\(B=0.2, D=0.4, N=120\)}
\label{fig:I(s)-b0.2n120}
\end{figure}

\begin{figure}
  \includegraphics{"B0.63 D0.4 N260/I(s)"}  % chktex 36
  \caption{\(B=0.63, D=0.4, N=260\)}
\label{fig:I(s)-b0.63n260}
\end{figure}

\end{document}
