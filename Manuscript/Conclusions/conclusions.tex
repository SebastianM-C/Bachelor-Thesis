\documentclass[../thesis.tex]{subfiles}

%!TeX spellcheck = en-GB

\begin{document}

\chapter{Conclusions}

In this thesis we have studied the connection between the classical phase space structures
and the spectra of the corresponding quantum system. The basic question was to
explore how the relative partition between tori and chaotic trajectories in the
classical phase space is reflected in the statistics of the energy levels of
the quantum analogue.

Starting with a Hamiltonian that describes the dynamics of the nuclear surface
in terms of quadrupole vibrations, we obtained the energy spectra through
numerical diagonalisation. The Hamiltonian depends on parameters, which at
classical level induce the transition from an integrable system to a non-integrable
one. As a consequence, in agreement with the K.A.M. theorem, the phase space
starts to be populated with chaotic trajectories which escape from the tori,
filling a 3 dimensional volume. We observed that the relative weights of the
tori volumes in the phase space depends in a non-trivial way with the energy
and the control parameter for non-integrability. Starting from this observation
we investigated systematically the properties of the energy spectra as a function
of the same control parameter and in different energy ranges. Specifically, we
focused on the nearest neighbour distribution and compared it with the predictions
of the Poisson and Wigner distributions.

We observed that for all values of the control parameter, the distribution manifests
deviations from the expected Wigner distribution characteristic to chaotic systems.
We assumed that this effects are related to the significant influence of the tori
at the classical level. Therefore, we proposed a distribution which is a linear
superposition of the Poisson and Wigner distributions. Then, the coefficient
factorising the Poisson distribution will be a measure of closeness to an
integrable behaviour. Indeed, if this coefficient is equal to unity, the distribution
becomes Poissonian, while when it goes to zero, it transforms Wigner one.

We noticed that when we rise the energy range for the analysis of the level
distribution, this coefficient increases, reflecting the observed behaviour
in classical phase space. In other words, the global structure of the classical
phase space for an energy interval is reflected by the superposition fitting coefficient.
More precisely, we observed that the value of this coefficient increases with
the considered energy interval, while in the classical system, as we
increase the energy, the volume of the regular trajectories confined on tori
increases.

Furthermore, we obtained a non-trivial dependence of the superposition coefficient
on the non-integrability parameter. While a greater value of this parameter is expected
to make the system more chaotic, and consequently the level distribution gets closer
to the Wigner one, from our analysis resulted a more intricate behaviour.
The superposition coefficient decreases and increases periodically when the
non-integrability parameter changes. This periodic shape shows that for specific values
of the control parameter, the system will manifest a more robust integrable-like
behaviour. Further work is required to better understand this aspect.

\end{document}
