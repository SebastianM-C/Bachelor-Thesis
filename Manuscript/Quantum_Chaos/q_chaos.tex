\documentclass[../thesis.tex]{subfiles}

%!TeX spellcheck = en-GB

\theoremstyle{definition}
\newtheorem*{def*}{Definition}

\begin{document}

\chapter{Quantum Chaos}
\label{chap:quantum-chaos}

\section{Fundamental notions}

In this section will briefly present the basic principles of quantum mechanics
and introduce the concepts which are required for the characterisation of
energy spectra.
We begin with the concept of the physical state.
A \emph{physical state} contains all the information that can be learned
about the system.
We associate to our physical system a \emph{Hilbert space}, denoted with
\(\mathbb{H}\), which will contain all the possible states of the system.
A vector in the Hilbert space will describe the state of the system
(more rigorously only the direction of the vector will describe the state
since by convention any two vectors that differ only by a constant describe
the same physical state). In the Dirac formalism~\cite{Dirac1967} the vectors in the Hilbert
space corresponding to physical states are represented by \emph{ket vectors},
denoted with \(\ket{\Psi}\). The elements of the dual of the Hilbert space
are called \emph{bra vectors} and are denoted with \(\bra{\Psi}\). Due to the
isomorphism between the Hilbert space and its dual, the bra vectors equally
describe the physical state and the correspondence between the ket vectors and
the bra vectors is given by dual conjugation.

\subsubsection{The observables postulate}

In quantum mechanics the properties of the physical system, called \emph{observables}
are described by linear hermitian operators.
As a result of a measurement, the obtained values of the observable are among
the eigenvalues of the associated hermitian operator.

\subsubsection{The measurement postulate}

Let us consider a physical system in a state described by the state vector
\(\ket{\Psi} \in \mathbb{H}\) and an observable described by the linear hermitian
operator \(A\). If a measurement of the observable is performed, then following
the measurement the system will jump in an uncontrollable manner in one of the
eigenstates of the operator associated with the measured observable and the result
of the measurement will be given by the corresponding eigenvalue. The probability
of obtaining a given eigenvalue is given by the square of the absolute value of
the scalar product between the final state associated with the given eigenvalue
and the initial state.

% {\color{red}discrete / continuous discussion?}

\subsubsection{Fundamental commutation relations}

In analogy with classical mechanics, a quantum version of the Poisson
brackets can be defined, postulating that they have the same properties
as the classical Poisson brackets, namely anti-symmetry, linearity, the product
rule and the Jacobi identity. These properties uniquely define the form
of the quantum Poisson brackets the commutator of two observables,
\(\comm{A}{B} = AB - BA\), as being proportional to the quantum Poisson brackets
\[
  \comm{A}{B} = \ii \hbar {\{A,B\}}_{QM}.
\]
Thus we obtain the fundamental commutation relations in quantum mechanics
\begin{align*}
  \comm{Q_i}{Q_j} &= 0 \\
  \comm{P_i}{P_j} &= 0 \\
  \comm{Q_i}{P_j} &= \ii \hbar \,\delta_{i,j}.
\end{align*}

\subsubsection{Time evolution postulate}

In quantum mechanics time is just a parameter and not an observable.
The time evolution can be postulated in different ways.
If we consider the states as time dependent entities and observables as time
independent ones, we can view the time
evolution as a temporal displacement, similar to a spatial displacement.
Then we can use the time evolution operator identity
\[
  \ii \hbar \pdv{t} U(t,t_0) = H U(t,t_0)
\]
to derive the Schrödinger equation:
\[
  \ii \hbar \pdv{t} \ket{\Psi} = H \ket{\Psi}.
\]
If we consider the states as time independent and the observables as time dependent,
then the time evolution of the observables will be given by
\[
  \ii \hbar \dv{t} A(t) = [A(t), H].
\]

We can observe the link between this equation and Hamilton's equations is given
by the correspondence between the commutator in quantum mechanics and the Poisson
bracket in classical mechanics, as mentioned in the previous chapter.

The first approach is called the Schrödinger picture of quantum mechanics, while
the second is called the Heisenberg picture. There is also another formulation,
namely the interaction picture also named the Dirac picture.
In the following we will only consider the Schrödinger picture.

In the particular case when the Hamiltonian is time-independent, Schrödinger's equation
reduces to
\[
  H \ket{\Psi} = E \ket{\Psi},
\]
also called the time-independent Schrödinger equation and the time evolution of
the state is given by
\[
  \ket{E(t)} = \ee^{-\frac{\ii}{\hbar} (t-t_0) H} \ket{E(t_0)}
             = \ee^{-\frac{\ii}{\hbar} (t-t_0) E} \ket{E}.
\]

Thus if the system is initially in an energy eigenstate, it will remain in the
same state, having at most a phase modulation. Such states are called
\emph{stationary states}.

\section{From symmetry to degeneracy}

Let us consider the above mentioned time-independent case. If we suppose that
the energy spectrum is discrete,
\[
  H \ket{\Psi_i} = E_i \ket{\Psi_i}.
\]
If for an eigenvalue \(E_i\) repeats itself for different eigenstates
\(\ket{\Psi_i}\) we have what is called a \emph{degeneracy}. In order to emphasise
this, we can use a second index for the eigenvectors, \(j=1,\dotsc,m\), where
\(m\) is the number of times the eigenvalue repeats.

Thus for the \(i\)-th eigenvalue,
\[
  H \ket{\Psi_{i,j}} = E_i \ket{\Psi_{i,j}}.
\]

If the Hamiltonian is invariant to a set of unitary transformations,
\(T^\dagger H T = H\), then this set forms a group since the transformation
given by \(T T'\), with \(T'\) an arbitrary transformation from the set,
will also leave the Hamiltonian invariant.
Since symmetry operations form a group, the unitary operators which correspond to the
symmetries of the Hamiltonian will form a group to which the Hamiltonian is invariant.
Thus,
\[
  T H = T T^\dagger H T = T T^{-1} H T = H T,
\]
or \(\comm{T}{H} = 0\). We can once again observe how classical mechanics and
quantum mechanics are linked through the Poisson bracket-commutator structure.
In classical mechanics Noether's theorem asserts that to any symmetry of the action
will correspond a constant of the motion. This will have a vanishing Poisson bracket
with the Hamiltonian. For example, the invariance to rotations around a given
axis will lead to angular momentum conservation.

Since the operators commute, they share a common set of eigenvectors
\[
  H (T \ket{\Psi_i}) = H T \ket{\Psi_i} = T H \ket{\Psi_i} = T E_i \ket{\Psi_i}
  = E_i (T \ket{\Psi_i}).
\]
Thus if the Hamiltonian is invariant to the set of unitary transformations
\( \{T_j\} \), with \(j=1,\dotsc,m\)
\[
  T_j H \ket{\Psi_i} = T E_i \ket{\Psi_i} = E_i (T \ket{\Psi_i}) = E_i \ket{\Psi_{i,j}}
\]
and we will not change the value of the eigenvalue \(E_i\) by applying \(T_j\),
we only transform the eigenstates \(\ket{\Psi_i}\) to a linear combination
\(\ket{\Psi_{i,j}}\).


\section{Level repulsion}

As is the case with classical mechanics there are few situations in which the
analytic solution is known. In order to find approximate solutions we can use
perturbation theory if we can assume that the Hamiltonian can be viewed as to
have a part for which the solutions are known and another part that can
be considered a small perturbation. The following concepts are mainly following
the notations from~\cite{Sakurai2011} and the lecture notes~\cite{Baran,Zus}.

In the framework of time-independent perturbation theory, we want to obtain
an approximate solution to the problem
\begin{equation}
  H \ket{\Psi^{[j]}} = E_j \ket{\Psi^{[j]}}
\label{eq:q-pert-th-ex-pr}
\end{equation}
and we consider that the Hamiltonian can be written as \(H = H_0 + \lambda V\),
where the solution to the unperturbed problem
\[
  H_0 \ket{\Psi^0_{n\alpha}} = E_n^0 \ket{\Psi^0_{n\alpha}}
\]
is known and \( \alpha \) gives the degeneracy of the \(n\)-th level.

Since the unperturbed eigenvectors form a basis in the Hilbert space, we can
expand the perturbed eigenvectors in that basis.
\[
  \ket{\Psi^{[j]}} = \sum_{m,\beta} c_{m,\beta}^{[j]} \ket{\Psi_{m,\beta}^0}.
\]
Inserting into eq.~\eqref{eq:q-pert-th-ex-pr} we obtain
\[
  (H_0 + \lambda V) \sum_{m,\beta} c_{m,\beta}^{[j]} \ket{\Psi_{m,\beta}^0}
  = \sum_{m,\beta} E_j\, c_{m,\beta}^{[j]} \ket{\Psi_{m,\beta}^0}.
\]
Using the solution to the unperturbed problem the above equation becomes
\[
\sum_{m,\beta} (E_m^0 + \lambda V)  c_{m,\beta}^{[j]} \ket{\Psi_{m,\beta}^0}
= \sum_{m,\beta} E_j\, c_{m,\beta}^{[j]} \ket{\Psi_{m,\beta}^0}.
\]

To simplify the discussion we will now consider what happens with two unperturbed
states \(\ket{\Psi_1^0}\) with the energy \(E_1^0\) and \(\ket{\Psi_2^0}\) with
the energy \(E_2^0\) when \(\lambda=1\).

\[
\sum_{m,\beta} (E_m^0 + V)  c_{m}^{[j]} \ket{\Psi_{m}^0}
= \sum_{m=1,2} E_j\, c_{m}^{[j]} \ket{\Psi_{m}^0}
.\]

By taking the scalar product with the state \(\bra{\Psi_{k}^0}\) and using the
orthogonality relation
\(\braket{\Psi_{k}^0}{\Psi_{m}^0} = \delta_{k,m}\)
we obtain
\[
E_k^0\, c_{k}^{[j]} + \sum_{m=1,2} \bra{\Psi_{k}^0} V \ket{\Psi_{m}^0} c_{m}^{[j]}
= E_j\, c_{k}^{[j]},
\]
or
\[
  c_{k}^{[j]} \left( E_k^0 - E_j + V_{k,k} \right) + \sum_{m \neq k} V_{k,m}\, c_{m}^{[j]} = 0,
\]
where \(\bra{\Psi_{k}^0} V \ket{\Psi_{m}^0} \equiv V_{k,m}\).

The above equation becomes
\begin{align*}
  (E_1^0 - E_j + V_{1,1})\, c_1^{[j]} + V_{1,2}\, c_2^{[j]} = 0, \text{ for } k=1 \\
  V_{2,1}\, c_1^{[j]} + (E_2 - E_j + V_{2,2})\, c_2^{[j]} = 0, \text{ for } k=2.
\end{align*}
We use the following notations:
\[
  H_{1,1} \equiv E_1^0 + V_{1,1}, H_{2,2} \equiv E_2^0 + V_{2,2}, H_{1,2} \equiv V_{1,2},
  H_{2,1} \equiv V_{2,1}, \delta \equiv H_{1,1} - H_{2,2}, \tan{\beta} \equiv \frac{2\abs{H_{1,2}}}{\delta}.
\]

The above system of equations has non-trivial solutions if the determinant vanishes.
\[
  (H_{1,1} - E_j)(H_{2,2} - E_j) - H_{1,2} H_{2,1} = 0
\]
or
\[
  E_j^2 - (H_{1,1} + H_{2,2}) E_j + H_{1,1} H_{2,2} - H_{1,2} H_{2,1} = 0.
\]
This equation has the solutions
\begin{align}
  E_j &= \frac{(H_{1,1} + H_{2,2}) \pm
         \sqrt{ {(H_{1,1} + H_{2,2})}^2 - 4(H_{1,1} H_{2,2} - H_{1,2} H_{2,1})}}{2} \\
      &= \frac{H_{1,1} + H_{2,2}}{2} \pm \frac{1}{2} \sqrt{\delta^2 + 4 \abs{H_{1,2}}^2}.
\label{eq:lvl-repulsion-e}
\end{align}

Since \((H_{1,1} - E_j)\, c_1^{[j]} + H_{1,2}\, c_2^{[j]} = 0\),
\[
  \frac{c_1^{[j]}}{c_2^{[j]}} = \frac{H_{1,2}}{E_j - H_{1,1}}.
\]
If we rewrite eq.~\eqref{eq:lvl-repulsion-e} as
\[
  E_j = \frac{1}{2} \left[ H_{1,1} + H_{2,2} \mp (H_{2,2} - H_{1,1})
        \sqrt{1 + \frac{4\abs{H_{1,2}}^2}{\delta^2}}\right]
\]
we obtain
\[
  \frac{c_1^{[j]}}{c_2^{[j]}} = \frac{2 H_{1,2}}{H_{2,2} - H_{1,1}}
  \left[ 1 \mp \sqrt{1 + \frac{4\abs{H_{1,2}}^2}{\delta^2}} \right]^{-1}
  = -\tan{\beta}\left(1 \mp \sqrt{1+\tan^2\beta}\right).
\]
This ratio can be also expressed as
\begin{align*}
  \frac{c_1^{[j]}}{c_2^{[j]}} &= \frac{-\tan{\beta}}{1 \mp \frac{1}{\cos{\beta}}}
  = \frac{-\sin{\beta}}{\cos{\beta} \mp 1} \\
  &= \frac{-2\sin{\frac{\beta}{2}} \cos{\frac{\beta}{2}}}
      {\cos^2{\frac{\beta}{2}} - \sin^2{\frac{\beta}{2}} \mp
        \left(\cos^2{\frac{\beta}{2}} + \sin^2{\frac{\beta}{2}}\right)}
  =
  \begin{cases}
    \cot{\frac{\beta}{2}} \\
    -\tan{\frac{\beta}{2}}
  \end{cases}.
\end{align*}
Thus,
\begin{align*}
  \ket{\Psi^{[1]}} &= \cos{\frac{\beta}{2}} \ket{\Psi_1^0} + \sin{\frac{\beta}{2}} \ket{\Psi_2^0} \\
  \ket{\Psi^{[2]}} &= -\sin{\frac{\beta}{2}} \ket{\Psi_1^0} + \cos{\frac{\beta}{2}} \ket{\Psi_2^0}.
\end{align*}

If the matrix elements of the interaction which mix distinct states are relatively small,
that is \(\abs{H_{1,2}} \ll \delta \), \(\beta \simeq 0\)
\[
  E_{1,2} = \frac{H_{1,1} + H_{2,2}}{2} \pm
            \frac{\delta}{2} \sqrt{1 + \frac{4 \abs{H_{1,2}}^2}{\delta^2}}
\]
Thus
\[
  E_1 \simeq H_{1,1} + \frac{\abs{H_{1,2}}^2}{\delta}
\]
and
\[
  E_2 \simeq H_{2,2} - \frac{\abs{H_{1,2}}^2}{\delta}.
\]
and the perturbed energy levels are close to the unperturbed ones.
The states will also be approximatively the unperturbed ones
\begin{align*}
  \ket{\Psi^{[1]}} &\simeq \ket{\Psi_1^0} \\
  \ket{\Psi^{[2]}} &\simeq \ket{\Psi_2^0}.
\end{align*}

If on the other hand, the matrix elements levels which mix distinct states are
relatively strong, that is \(\abs{H_{1,2}} \gg \delta \), \(\beta \simeq \frac{\pi}{2}\)
\[
  E_{1,2} = \frac{H_{1,1} + H_{2,2}}{2} \pm
            \sqrt{\frac{\delta^2}{4} + \abs{H_{1,2}}^2}
          \simeq \frac{H_{1,1} + H_{2,2}}{2} \pm
          \left(\abs{H_{1,2}} + \frac{\delta^2}{8\abs{H_{1,2}}}\right)
\]
and
\begin{align*}
  \ket{\Psi^{[1]}} &\simeq \frac{\sqrt{2}}{2} \ket{\Psi_1^0} + \frac{\sqrt{2}}{2} \ket{\Psi_2^0} \\
  \ket{\Psi^{[2]}} &\simeq -\frac{\sqrt{2}}{2} \ket{\Psi_1^0} + \frac{\sqrt{2}}{2} \ket{\Psi_2^0}.
\end{align*}
In this case we observe that if \(\delta \simeq 0\),
then \(E_1 - E_2 \simeq 2\abs{H_{1,2}}\). If the unperturbed energy levels were
degenerated, then the perturbed energy levels will not remain so. The removal
of the degeneracy is called \emph{level repulsion}.

We can illustrate this phenomena by considering an arbitrary \(2 \cross 2\)
hermitian matrix and computing its eigenvalues.
\[
  H = \begin{pmatrix}
  H_{1,1}   & H_{1,2} \\
  H_{1,2}^* & H_{2,2}
  \end{pmatrix}
\]
The eigenvalues will be given by eq.~\eqref{eq:lvl-repulsion-e}.

The first case, \(\abs{H_{1,2}} \ll \delta \), can be viewed as the case when
the off-diagonal elements are small and the matrix can be approximated with
a diagonal matrix. In this case we expect that the energy levels will be close
to the diagonal levels or the unperturbed levels. The perturbed eigenstates
will also be approximatively equal with the unperturbed eigenstates.

The level repulsion emphasised by the second case, \(\abs{H_{1,2}} \gg \delta \),
corresponds to the case when the off-diagonal elements are significant. When we
expand the perturbed eigenstates in the basis given by the unperturbed eigenstates
we will have significant components from each element in the basis and the new
states will be a mixture of the unperturbed states. We can say that the energy levels
are no longer independent. In the case when the unperturbed energy
levels are degenerated, the difference between the perturbed energy levels
is given by \(2 \abs{H_{1,2}}\), which is a measure of the mixing of the
unperturbed states.


\section{Probability notions}

\begin{def*}[Joint probability]
  Given two events $A$ and $B$, their joint probability, \(P(A \cap B)\), is
  the probability of the two events to occur simultaneously.
\end{def*}

\begin{def*}[Conditional probability]
  The conditional probability of an event $A$ given an event $B$ with \(P(B)>0\),
  denoted \(P(A\,|\,B)\) is given by
  \[
    P(A\,|\,B) = \frac{P(A \cap B)}{P(B)}.
  \]
\end{def*}

The joint probability of $A$ and $B$ can be expressed as
\(P(A \cap B) = P(A\,|\,B) P(B)\).

\begin{def*}[Continuous random variable]
  A continuous random variable is a function from the set of all outcomes to the
  set of real numbers \(X:\Omega \to \mathbb(R)\) such that
  \[
    P(a \leq X(\omega) \leq b) = \int_a^b f(x) \dd{x},
  \]
  where \(f(x) \geq 0\) and \(\int_{-\infty}^{+\infty}f(x)\dd{x}=1\).
\end{def*}
The above function \(f\) is called the \emph{probability density function}.
The probability for a continuous random variable $X$ to take a value in the
infinitesimal interval of length \(\dd{x}\) is given by
\begin{equation}
  \label{eq:prob-rv-in-int}
  P(x \in [x, x + \dd{x}]) = \int_x^{x+\dd{x}} f(z) \dd{z} \approx f(x)\dd{x}
\end{equation}


\section{Nearest neighbour distributions}

% \emph{Random matrices} are matrices which have random variables as elements,
% with their randomness is restricted by the symmetries of the whole matrix.

Nearest neighbour spacing distributions show how the differences
between consecutive energy levels fluctuate around the average.
In order to better understand this concept we shall begin with the simpler
case of real random numbers as presented in~\cite{Timberlake2006} and then
continue with a general case as in~\cite{Brody1981}.

\subsection{The nearest neighbour spacing distribution of random numbers}

We consider a sequence of uniformly distributed, ordered, real, random numbers.
We define the \emph{spacing} of an ordered sequence as the sequence of
differences between consecutive elements.
For an interval of length \(s, s>0\), we will denote with \( P(n \in s) \) the probability
for the interval to contain $n$ numbers and with \( P(n \in \dd{s} |\; m \in s) \)
the conditional probability for the interval of length \( \dd{s} \) to contain
$n$ numbers given that the interval of length $s$ contains $m$ numbers.

If $E$ is a given number in the sequence, we are interested in the probability
\( P(s)\dd{s} \) to have the next number between \( E+s \) and \( E+s+\dd{s} \).
Since we are interested in the next number after $E$, we know that in the
interval of length $s$ there is no other number and the next number is somewhere
in the infinitesimal interval \(\dd{s}\). Thus the joint probability of
the events \(1 \in \dd{s}\) and \(0 \in s\) is given by:
\begin{equation}
  \label{eq:jpr-next}
  P(s)\dd{s} = P(1 \in \dd{s} |\; 0 \in s) P(0 \in s).
\end{equation}

Since random numbers are not correlated, the probability of a random number to be found
in the interval \( \dd{s} \) does not depend on the number of random numbers in $s$, so
\[
  P(1 \in \dd{s} |\; 0 \in s) = P(1 \in \dd{s}).
\]

The random numbers are uniformly distributed, so their probability density function
\(f\) is a constant. Hence, according to eq.~\eqref{eq:prob-rv-in-int}, the probability
of finding a number in the interval of length \( \dd{s} \) is given by
\[
  P(1 \in \dd{s}) \equiv P(1 \in [0, 0+\dd{s}]) \approx f(s) \dd{s} \sim \dd{s}.
\]
If we denote the constant probability density function with $a$
\[
  P(s)\dd{s} = a \dd{s} P(0 \in s).
\]
\( P(0 \in s) \) can be expressed using the complementary probability as
\( {1 - \int_0^s P(s') \dd{s'}} \). Now we can express \( P(s)\dd{s} \) as follows:
\[
  P(s)\dd{s} = a \dd{s} \left( 1 - \int_0^s P(s') \dd{s'} \right).
\]

In order to differentiate with respect to $s$, we will use the Leibniz rule
for differentiating integrals, namely
\begin{equation}
  \label{eq:leibnitz}
  \dv{x} \int\limits_{G(x)}^{H(x)} F(x, t) \dd{t} = \int\limits_{G(x)}^{H(x)} \pdv{F}{x} \dd{t}
  + F(x, H(x))\, \dv{H}{x} - F(x, G(x))\, \dv{G}{x}
\end{equation}

Using this rule, we obtain
\[
  \dv{s}P(s) = -aP(s).
\]
This differential equation can be solved by separation of variables, yielding
\[
  P(s) = \mathcal{C} \ee^{-as}
\]
In order to determine the constant \(\mathcal{C}\), we use the normalisation condition
for the probability density function
\[
  \int_{-\infty}^{\infty} P(s) \dd{s} = 1
\]
Since \(s>0\), this reduces to
\[
  \int_{0}^{\infty} P(s) \dd{s} = \int_{0}^{\infty} \mathcal{C} \ee^{-as} \dd{s}
  = -\frac{\mathcal{C}}{a} \eval{\ee^{-as}}_0^{\infty} = \frac{\mathcal{C}}{a}
\]
Thus \(\mathcal{C} = a\) and \(P(s) = a \ee^{-as}\).
We can further simplify the formula if we set that the average spacing to unity.
The average spacing is given by
\[
  \mean{s} = \int_{0}^{\infty} s P(s) \dd{s}
  = -\int_{0}^{\infty} s \dv{s} \left( \ee^{-as} \right) \dd{s}
  = \int_{0}^{\infty} \ee^{-as} \dd{s} = \frac{1}{a},
\]
so setting it to unity results in \(a=1\).
Thus the probability density function becomes
\begin{equation}
  \label{eq:poisson-dist}
  P(s) = \ee^{-s}.
\end{equation}
This function is known as the \emph{Poisson distribution}.

\subsection{The Wigner distribution}

We will now consider a more complicated situation, by considering the
probability density function for the sequence of random numbers to be arbitrary.

We can start from eq.~\eqref{eq:jpr-next} since the discussion up to that point
did not include any details related to the distribution of the random numbers.

In this case \(P(1 \in \dd{s} |\; 0 \in s) = f_{1,0}(s) \dd{s}\), where
\(f_{n,m}(s)\) is function which describes how the probability of having $n$
numbers in \(\dd{s}\) is influenced by the $m$ numbers in $s$.
Thus
\[
  P(s)\dd{s}=f_{1,0}(s)\dd{s} \left( 1 - \int_0^s P(s') \dd{s'} \right)
\]
By solving
% {\color{red}\large????}
this integral equation, we obtain the solution
\[
  P(s) = \mathcal{C} f_{1,0}(s) \exp(-\int_0^s f_{1,0}(x) \dd{x})
\]
We observe that if we take the probability density function constant,
\(f_{1,0}(s) = \frac{1}{a}\), we obtain the above case of the Poisson distribution.
For a
% {\color{red}linear (why?)}
probability density function,
\(f_{1,0}(s) = \alpha s\) we obtain
\[
  P(s) = \mathcal{C} \alpha s \exp(-\alpha \frac{s^2}{2})
\]
From the normalisation condition we obtain
\[
  \int_{0}^{\infty} P(s) \dd{s} = \mathcal{C} = 1
\]
The average spacing is given by
\[
  \mean{s} = \int_{0}^{\infty} s P(s) \dd{s}
  = \alpha \int_{0}^{\infty} s^2 \exp(-\alpha \frac{s^2}{2}) \dd{s}
  = \frac{1}{\sqrt{\alpha}} \sqrt{\frac{\pi}{2}}.
\]
If we set the average spacing to unity, we obtain \(\alpha = \frac{\pi}{2}\)
and
\[
  P(s) = \frac{\pi}{2} s \exp(-\frac{\pi}{4} s^2)
\]
This function is known as the \emph{Wigner distribution}.

\section{From Classical Chaos to Quantum Chaos}

The classical concept of sensitivity to initial conditions loses its meaning
in the quantum realm since the trajectory cannot be defined due to
Heisenberg's uncertainty principle. However, there are some other
ways in which we can link classically chaotic dynamics to quantum features.
These bridges between classical mechanics and quantum
mechanics allows us to give a meaning to quantum chaos~\cite{Berry1989}.

% In the terms of classical mechanics, such as the sensitivity to initial conditions
% quantum chaos does not exist~\cite{Berry1989} because of the Heisenberg's uncertainty principle,
% the absence of trajectories and many other reasons. In turn, there are some other
% ways in which we can link classically chaotic systems to their
% quantum counterparts. These bridges between classical mechanics and quantum
% mechanics allows us to give a meaning to quantum chaos.
% {\color{red} (Berry $\to$ quantum chaology)}

Specifically, there are two important conjectures that allow us to connect
classical systems and quantum systems as mentioned above.

\subsubsection{The Berry-Tabor conjecture}% {\color{red}(or theorem?)}}

This conjecture states that the quantum counterpart of a classically integrable
system has a Poissonian nearest neighbour distribution.

\subsubsection{The Bohigas-Gianoni-Schmit conjecture}

This conjecture states that the nearest neighbour distribution of a quantum system
with a classically chaotic counterpart is given by the Wigner distribution.



\end{document}
