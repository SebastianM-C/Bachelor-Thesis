\documentclass[../thesis.tex]{subfiles}

\begin{document}

\chapter{Quantum Chaos}

\section{Level repulsion}

\section{Random matrix theory}

{\color{red}define random matrices}

Nearest neighbour spacing distributions show how the differences
between consecutive energy levels fluctuate around the average.
In order to understand better this concept we shall begin with the simpler
case of real random numbers.

\subsection{The nearest neighbour spacing distribution for real random numbers}

We consider a sequence of uniformly distributed, ordered, real, random numbers.
Let $E$ represent a number in the sequence.
The probability \( P(s)\dd{s} \) to have the next number between \( E+s \) and
\( E+s+\dd{s} \) is given by:
\[
  P(s)\dd{s} = P(1 \in \dd{s} |\, 0 \in s) P(0 \in s),
\]
where \( P(n \in s) \) represents the probability for $s$ to contain $n$ numbers and
\( P(n \in \dd{s} |\, m \in s) \) is the \emph{conditioned} probability for
the interval of length \( \dd{s} \) to contain $n$ numbers when the interval of
length $s$ contains $m$ numbers.

Since random numbers are not correlated, the probability of a random number to be found
in the interval \( \dd{s} \) does not depend on the number of random numbers in $s$, so
\[
  P(1 \in \dd{s} |\, 0 \in s) = P(1 \in \dd{s}).
\]

Since the random numbers are uniformly distributed, the probability {\color{red} (density?)} of finding a number
in the interval \( \dd{s} \) is constant. We denote this constant with $a$. Thus
\[
  P(s)\dd{s} = a \,{\color{red}\dd{s}} P(0 \in s).
\]
\( P(0 \in s) \) can be expressed using the complementary probability as
\( {1 - \int_s^\infty P(s') \dd{s'}} \). Now we can express \( P(s)\dd{s} \) as follows:
\[
  P(s)\dd{s} = a \,{\color{red}\dd{s}} \left( 1 - \int_s^\infty P(s') \dd{s'} \right).
\]

Using the Leibniz rule for differentiating integrals,
\[
  \dv{x} \int\limits_{G(x)}^{H(x)} F(x, t) \dd{t} = \int\limits_{G(x)}^{H(x)} \pdv{F}{x} \dd{t}
  + F(x, H(x))\, \dv{H}{x} - F(x, G(x))\, \dv{G}{x}
\]
we obtain
\[
  {\color{red}\dv{s}}P(s) = -aP(s)
\]
when differentiating with respect to $s$.

\end{document}
